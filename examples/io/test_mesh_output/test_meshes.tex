\documentclass[a4wide]{article}
\usepackage{etex}
\usepackage{tikz}
\usepackage{graphicx}

\title{LehrFEM++ Test Meshes}

\date{\today}

\begin{document}

\maketitle

\noindent
\LaTeX document including LehrFEM++ TikZ output generated by
\begin{quote}
  \texttt{lf::mesh::utils::writeTikZ()}
\end{quote}
in the LehrFEM++ demo code
\begin{quote}
  \texttt{lf.examples.mesh.test\_mesh\_output\_demo}.
\end{quote}
That code creates
files \texttt{test\_mesh\_X.tex}, where \texttt{X} stands for the
selector value corresponding to the test mesh.

\section*{Test Mesh \#0}

\resizebox{\linewidth}{!}{\input{test_mesh_0}}

\section*{Test Mesh \#1}

\resizebox{\linewidth}{!}{\input{test_mesh_1}}

\section*{Test Mesh \#2}

\resizebox{\linewidth}{!}{\input{test_mesh_2}}

\section*{Test Mesh \#3}

\resizebox{\linewidth}{!}{\input{test_mesh_3}}

\section*{Test Mesh \#4}

\resizebox{\linewidth}{!}{\input{test_mesh_4}}

\section*{Test Mesh \#5}

\resizebox{\linewidth}{!}{\input{test_mesh_5}}

\section*{Test Mesh \#6}

\resizebox{\linewidth}{!}{\input{test_mesh_6}}

\section*{Test Mesh \#7}

\resizebox{\linewidth}{!}{\input{test_mesh_7}}

\end{document}

%%% Local Variables:
%%% mode: latex
%%% TeX-master: t
%%% End:
